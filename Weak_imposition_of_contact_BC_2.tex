\documentclass{article}
\usepackage[utf8]{inputenc}
\usepackage{amsmath,amsfonts,amssymb,bm}
\usepackage{graphicx,xcolor}
\usepackage{hyperref}
\newcommand{\eps}{\varepsilon}
\newcommand{\lam}{\lambda}
\newcommand{\beq}{\begin{equation}}
\newcommand{\eeq}{\end{equation}}
\newcommand{\bea}{\begin{eqnarray}}
\newcommand{\eea}{\end{eqnarray}}

\newcommand{\dx}{{\rm d}x}
\newcommand{\ints}{\int_{\mathcal{S}}}
\newcommand{\ts}{\tilde{s}}

\newcommand{\Iom}{I_{\omega}}
\newcommand{\Ik}{I_{\rm k}}
\newcommand{\mxc}{m_{x{\rm c}}}
\newcommand{\mb}{m_{\rm b}}
\newcommand{\dd}{\,{\rm d}}
\newcommand{\ignore}[1]{}
\begin{document}

\pagestyle{empty}


 


\definecolor{r}{rgb}{0.9,0,0}
\begin{Large}

\begin{center}

\textbf{CZECH TECHNICAL UNIVERSITY 
\\
IN PRAGUE
\\
FACULTY OF CIVIL ENGINEERING}

\vspace{1cm}

\textbf{DEPARTMENT OF MECHANICS}

\end{center}


\vspace{1cm}

\begin{figure}[h]

\centering
  
\includegraphics[scale=0.4]{lev.jpg}

\end{figure} 

\begin{center}

\vspace{1cm}

\textbf{HIGH PERFORMANCE SOLVERS FOR CONTACT MECHANICS}

\vspace{1cm}

%\textbf{BAKALÁŘSKÁ PRÁCE}

\end{center}

\begin{flushleft}


\vspace{1cm}

\textbf{Author:}\hspace*{\fill}
\textbf{...}

\vspace{0,5cm}

\textbf{Supervisor:} \hspace*{\fill}
\textbf{...}




\vspace{1cm}

\end{flushleft}

\begin{center}

\textbf{2019/2020} 

\end{center}

\end{Large}

%\newpage

%\vspace*{\fill}
%\begin{Large}

%\noindent \textbf{Čestné prohlášení}

%\end{Large}

%\vspace{0.5cm}

%Prohlašuji, že jsem svou práci vypracoval samostatně, použil jsem pouze podklady %(literaturu, SW atd.) uvedené v přiloženém seznamu. 

%\vspace{2cm}

%\noindent V ………… dne ………………… podpis: ……………………………

\newpage

\vspace*{\fill}
\begin{Large}

\noindent \textbf{Declaration of Authorship}

\end{Large}

\vspace{0.5cm}

\noindent I declare that this diploma thesis has been carried out by me and only with the use of materials that are stated in the literature sources. 

\vspace{2cm}

\noindent In ………… on ………………… signature: ……………………………

\newpage

\vspace*{\fill}

\begin{Large}

\noindent \textbf{Acknowledgements}

\end{Large}

\vspace{0.5cm}

\begin{normalsize}
... 
\end{normalsize}

\newpage

\section*{Anotace}
%\addcontentsline{toc}{chapter}{Abstrakt}
... 

\section*{Klíčová slova}

... 

\section*{Annotation}
%\addcontentsline{toc}{chapter}{Abstract}

...    

\section*{Keywords}

...  

\newpage

\tableofcontents

\newpage

\pagestyle{plain}

\setcounter{page}{6}


\section{Introduction}
\noindent ...
\newpage
\section{Weak form of contact boundary value problem}
In this chapter we will present the mathematical formulation of problem of finite deformation of bodies which boundaries can come in contact with each other using continuum solid mechanics approach.
\\
\\
Before we will proceed to weak formulation for contact boundary value problem, we will, for the purpose of clarity, firstly consider the problem of finite deformation of system of bodies $ \Omega^{\gamma} $ under static loading. By adding the so called contact boundary conditions, we will obtain the the weak form of contact boundary value problem.
\\
\\
The concept of the contact boundary conditions will be specified later. Let us now only suggest, that, roughly said, contact boundary conditions describe the possibility of contact of bodies $ \Omega^{\gamma} $ between each other or even self-contact of different parts of boundary of each body $ \Omega^{\gamma} $ (though we will not consider the case of self-contact in this thesis). These conditions will be generally formulated in form of inequalities, as the part of the boundary of bodies $ \Omega^{\gamma} $, which is in contact, is not known in forward. This makes the contact problems generally nonlinear. Even if the equations describing contact problem are linear, the fact, that contact area is not known in advance, which leads to the inequality form of boundary conditions, eventually makes the contact problems nonlinear.   
\subsection{Weak form of problem of finite deformation of system of bodies $ \Omega^{\gamma} $ under static loading}
  

   We will consider motion and deformation of continuum deformable bodies $ \Omega^{\gamma} $, $\gamma = 1,2,...,n$ where integer $ n>0 $  is a number of deformable bodies which we consider. $\Omega^{\gamma}$ is a region in Euclidean space $\mathbb{E}^{3}$. In time, the material points of bodies $ \Omega^{\gamma} $ could change its position in space and we suppose, that for each body $ \Omega^{\gamma} $ there exists one-to-one mapping $\bm{\varphi}^{\gamma}(t) : \Omega^{\gamma} \longrightarrow \mathbb{E}^{3}$ describing that motion in time $t \in \mathbb{R}^{+}$. Initial (undeformed) position of material points of $ \Omega^{\gamma} $ is denoted by $ \mathbf{X} $. In time, this position changes to $ \mathbf{x} $, while it stands $\mathbf{x}=\bm{\varphi}^{\gamma}(\mathbf{X}, t) = \mathbf{X}+\mathbf{u}(\mathbf{X}, t)$. It is highly important, that we presume that for Jacobian $ J^{\gamma} $ of the mapping $\bm{\varphi}^{\gamma}(t)$ holds $J^{\gamma}=\operatorname{det} \mathbf{F^{\gamma}} > 0$, where $\mathbf{F}=\operatorname{Grad} \bm{\varphi}^{\gamma}(\mathbf{X}, t)$. $ \operatorname{Grad} $ is a gradient operator with respect to coordinates $ \mathbf{X} $. Let us denote on this place that operators with respect to coordinates $ \mathbf{X} $ are written with capital beginning letter (e.g. $ \operatorname{Grad} $) and operators with respect to coordinates $ \mathbf{x} $ are denoted by small letter (e.g. $ \operatorname{grad} $). The condition $J^{\gamma} > 0$ is necessary to obtain a deformation, which is physically meaningful ($ J^{\gamma} = 0 $ could be understood as collapse of infinitesimal volume to a point) and from the mathematical point of view it is necessary to ensure the existence of inverse mapping $ \bm{\varphi}^{-1,\gamma}(t) $ so that it holds $\mathbf{X}=\bm{\varphi}^{-1,\gamma}(\mathbf{x},t)$. By this condition we also ensure the regularity of $ \bm{\varphi}^{\gamma}(t) $. 
\\
\\
According to the theory of continuum solid mechanics, equilibrium of deformable body can be described as
\begin{equation}\label{1}
\operatorname{Div} \mathbf{P}+\rho_{0} \mathbf{\overline{b}}=\rho_{0} \dot{\mathbf{v}}
\end{equation}
where $ \mathbf{P} $ is so called first Piola–Kirchhoff stress tensor (see \cite[Chapter 2, p.28]{Wriggers}). $ \operatorname{Div} $ is the divergence operator. Term $ \rho_{0} \mathbf{\overline{b}} $ defines the volume load force, typically due to gravitation. Term $ \rho_{0} \dot{\mathbf{v}} $ denotes inertia force and $ \rho_{0} $ is a density in the initial configuration. 
\\
\\
We will now derive the weak form of equations describing the equilibrium of the considered bodies $ \Omega^{\gamma} $. Firstly, sheerly for the purpose of simplicity, we will consider only static loading of the body and so neglect the term $ \rho_{0} \dot{\mathbf{v}} $. Then we will multiply both sides of the equation (\ref{1}) by vector valued function $\mathbf{v}\in \mathbb{V}; \mathbb{V} =\left\{\mathbf{v} ; \mathbf{v}=\mathbf{0} \text { on } \Gamma_{\varphi}^{\gamma}\right\}$. $ \Gamma_{\varphi}^{\gamma} $ denotes the subset of boundary $ \Gamma^{\gamma} $, where $ \Gamma_{\varphi}^{\gamma} \subset \Gamma^{\gamma}$. On $ \Gamma_{\varphi}^{\gamma}$ the Dirichlet type boundary condition $ \bm{\varphi} = \overline{\bm{\varphi}} $ is prescribed. Let us remark, that  $\mathbf{v}$ is often called a virtual displacement or test function. After the integration over the volume of all considered deformable bodies $ \Omega^{\gamma} $, we obtain
\begin{equation}\label{2}
\sum^{n}_{\gamma = 1}\left(\int_{\Omega^{\gamma}} \operatorname{Div} \mathbf{P} \cdot \mathbf{v} \dd V+\int_{\Omega^{\gamma}} \rho_{0}\overline{\mathbf{b}} \cdot \mathbf{v} \dd V\right)=0
\end{equation}
Applying famous Gauss-Ostrogradsky theorem and multiplying whole equation by $-1$ gives us
\begin{equation}\label{3}
\sum^{n}_{\gamma = 1}\left(\int_{\Omega^{\gamma}} \mathbf{P} : \operatorname{Grad} \mathbf{v} \dd V-\int_{\Gamma_{\sigma}^{\gamma}} \overline{\mathbf{t}} \cdot \mathbf{v} \dd A-\int_{\Omega^{\gamma}} \rho_{0}\overline{\mathbf{b}} \cdot \mathbf{v} \dd V\right)=0
\end{equation}
$ \Gamma_{\sigma}^{\gamma} $ denotes again the subset of boundary $ \Gamma^{\gamma} $, so it holds $ \Gamma_{\sigma}^{\gamma} \subset \Gamma^{\gamma}$. On $ \Gamma_{\sigma}^{\gamma}$ the  Neumann type boundary condition $ \mathbf{N}\cdot\mathbf{P} = \overline{\mathbf{t}}$ is prescribed. $ \overline{\mathbf{t}} $ is surface load at given point $ \mathbf{X} $ and is generally not collinear with $  \mathbf{N} $, which is a normal to the surface $ \Gamma_{\sigma}^{\gamma}$ at considered point $ \mathbf{X} $.  
We require (\ref{3}) to be fulfilled for $\forall\mathbf{v}\in H; H =\left\{\mathbf{v} ; \mathbf{v}=\mathbf{0} \text { on } \Gamma_{\varphi}^{\gamma}\right\}$. As previously stated, we require following boundary conditions to hold
\beq\label{4}
\bm{\varphi} = \overline{\bm{\varphi}} \text{ on } \mathbf{\Gamma_{\varphi}^{\gamma}}
\eeq
\beq\label{5}
\mathbf{N}\cdot\mathbf{P} = \overline{\mathbf{t}} \text{ on } \mathbf{\Gamma_{\sigma}^{\gamma}}  
\eeq
This is a general continuum mechanics formulation for problem of finite deformation of bodies $ \Omega^{\gamma} $ under static loading. This problem, stated by equation (\ref{3}) and boundary conditions (\ref{4}) - (\ref{5}), demanded to be fulfilled $\forall\mathbf{v}\in \mathbb{V}; \mathbb{V} =\left\{\mathbf{v} ; \mathbf{v}=\mathbf{0} \text { on } \Gamma_{\varphi}^{\gamma}\right\}$ is often referred as "virtual work principle" and is generally valid for any considered material constitutive law, as it was derived solely on the basis of equations of equilibrium of the considered body and no specific constitutive law was presumed. Let us denote, that equation (\ref{3}) can be equivalently rewritten as
\begin{equation}\label{6}
\sum^{n}_{\gamma = 1}\left(\int_{\Omega^{\gamma}} \bm{\tau} : \operatorname{grad} \mathbf{v} \dd V-\int_{\Omega^{\gamma}} \rho_{0}\overline{\mathbf{b}} \cdot \mathbf{v} \dd V-\int_{\Gamma_{\sigma}^{\gamma}} \overline{\mathbf{t}} \cdot \mathbf{v} \dd A\right)=0
\end{equation}
where $\bm\tau$ is called Kirchhoff stress tensor (see \cite[Chapter 2, p.28]{Wriggers}) . Kirchhoff stress tensor $\bm\tau$ is connected with first Piola–Kirchhoff $\mathbf{P}$ threw relationship $ \bm{\tau} = \mathbf{P}\mathbf{F}^{T}  $ \cite[Chapter 5, p.97]{Wriggers}  , where $\mathbf{F}=\operatorname{Grad} \bm{\varphi}(\mathbf{X}, t)$. Let us also note, that operator $ \operatorname{Grad} $ changed to $ \operatorname{grad} $ in equation (\ref{6}). 
\\
\\
In case of hyperelastic materials (see \cite[Chapter 2, p.31-36]{Wriggers}), we can derive "virtual work principle" minimizing the total potential energy system of all bodies $ \Omega^{\gamma} $. Functional of total potential energy $\Pi(\bm{\varphi})$ is in this case given as (\cite[Chapter 5, p.98]{Wriggers})
\begin{equation}\label{7}
\Pi(\bm{\varphi}) = \sum^{n}_{\gamma = 1}\left(\int_{\Omega^{\gamma}} W(\mathbf{C})\dd V-\int_{\Omega^{\gamma}} \rho_{0}\overline{\mathbf{b}} \cdot \mathbf{\bm{\varphi}} \dd V-\int_{\Gamma_{\sigma}^{\gamma}} \overline{\mathbf{t}} \cdot \mathbf{\bm{\varphi}} \dd A\right)=0
\end{equation}
Here $  W(\mathbf{C}) $ is so called strain energy function. This functional generally depends on right Cauchy-Green strain tensor $ \mathbf{C}$. For details see \cite[Chapter 2, p.32-33]{Wriggers}.
Stationary point of this functional (which is expected to be a local minimum\footnote{Though this should be in ideal case mathematically verified.}) can be found requiring zero valued Gateaux derivatives $ \delta \Pi(\bm{\varphi},\mathbf{v}) $ at stationary point for all admissible $ \mathbf{v} $, which are to be specified later. This can be formalized as (see \cite[Chapter 2, p.31]{Wriggers})
\beq\label{8}
\delta \Pi(\bm{\varphi},\mathbf{v})=\left.\frac{\dd}{\dd \alpha} \Pi(\bm{\varphi}+\alpha \mathbf{v})\right|_{\alpha=0} = 0
\eeq
It can be shown, that condition (\ref{8}) leads to equation analogical to (\ref{3}) or (\ref{6}). However, the total potential energy functional does not necessarily have to exist, if we consider different than hyperelastic material constitutive law. For example for case of visco-elastic or plastic constitutive laws no such functional could be found . In these cases, it is important that we can still use "virtual work principle" which is independent of material constitutive law.
\subsection{Contact boundary conditions}
Now, to finish the weak formulation of contact boundary value problem, we will present the contact boundary conditions describing possibility of contact on the boundary $ \Gamma^{\gamma,C}$, where $ \Gamma^{\gamma,C} \subset \Gamma^{\gamma} $.
\\
\\
As was suggested before, the great problem of contact problems is, that we do not know in advance $ \Gamma^{\gamma,C}$, i.e. on which part of the boundary $ \Gamma^{\gamma} $, contact physically happens and where so the contact boundary conditions should be employed. This fact eventually leads to nonlinear nature of contact problems. During numerical solution procedures of contact problems, it is necessary to really detect the areas, which are in contact. In order to achieve that, the so called global and local search algorithms are employed (see \cite[Chapter 9]{Wriggers}). However, from the prospective of theoretical formulation of contact problems, we do not know $ \Gamma^{\gamma,C}$ and therefore the contact boundary conditions will be stated in form of inequalities. During numerical solution, as the contact part of the boundary is detected, these inequalities change to equalities. It will be obvious to the reader later in this sub-chapter, that these "incremental" equalities can be actually understood as Dirichlet boundary conditions \cite[Chapter 4, p.109]{Yastrebov}.
\\
\\
Generally, we have to state contact boundary conditions both for normal contact as well as for the tangential contact \cite[Chapter 3, p.46]{Wriggers}. In this thesis, sheerly for the purpose of clarity, we will limit ourselves to case of normal frictionless contact. Normal contact boundary condition is given by following inequality. It is also known as the inequality constraint or non-penetration condition \cite[Chapter 3, p.47]{Wriggers}  
\beq\label{9}
g_{n}=\left(\mathbf{x}^{S}-\overline{\mathbf{x}}^{M}\right) \cdot \mathbf{n}^{M} \geq 0
\eeq  
Meaning of this definition is, that non-physical penetration of one body into another is  prohibited. We can firstly observe, that the inequality is formulated with respect to current (deformed) configuration of the body $\Omega^{\gamma} $. In this definition, $ \mathbf{x}^{S} $ is arbitrarily chosen so called "slave" point from "slave" body $\Omega^{\gamma,S} $ and $\mathbf{x}^{M} $ is so called "master" point from "master" body $\Omega^{\gamma,M} $ (which is different from slave body). Master point is connected to slave point via the minimum distance problem (see (\ref{10})). $ \mathbf{n}^{M} $ is normal to the surface of the master body at point $\mathbf{x}^{M} $. If $ g_{n} > 0 $, the points are not in contact, if $ g_{n} = 0 $, points are exactly in contact and finally when $ g_{n} < 0 $, there is a non-physical penetration of one body to another, which we are trying to prevent.   
\\
\\
Now let us mention few words on "master-slave" approach. According to \cite[Chapter 4, p.45-46]{Izi}: "Historically, through the development of the currently popular finite element codes (ABAQUS, ANSYS, LS-DYNA, etc.) the "master-slave" approach has been proved to be the most efficient approach in computational contact mechanics." Generally say, this approach is introduced for the purpose of so called Closest Point Projection Procedure (CPP), which assigns to every slave point $ \mathbf{x}^{S} $ such a master point $ \overline{\mathbf{x}}^{M} $, that $ \hat{d} $ is minimized
\begin{equation}\label{10}
\hat{d}=\left\|\mathbf{x}^{S}-\overline{\mathbf{x}}^{M}\right\|=\min _{\mathbf{X}^{M} \subseteq \Gamma^{\gamma,M}}\left\|\mathbf{x}^{S}-\mathbf{x}^{M}\right\|
\end{equation}
It can be shown (see \cite[Chapter 3, p.47]{Wriggers} for more detail), that this definition is equivalent to such a choice of $ \overline{\mathbf{x}}^{M} $ to $ \mathbf{x}^{S} $, that $ \mathbf{x}^{S} $ is a normal projection of $ \overline{\mathbf{x}}^{M} $ to $ \Gamma^{\gamma,S}$, where the normal is chosen with respect to $ \Gamma^{\gamma,M}$ at point $ \overline{\mathbf{x}}^{M} $. We see, that "master-slave" approach requires us to know $ \mathbf{n}^{M} $ and so requires us to know the geometry of surface $ \Gamma^{\gamma,M}$. That will be usually described using convective coordinates (see \cite[Appendix B]{Wriggers} for more detail) as they provide easier description of surface $ \Gamma^{\gamma,M}$ than the global Cartesian coordinate system does\footnote{According to \cite[Appendix B, p.405]{Wriggers}, convective coordinates can be understood as: "Coordinates, which are attached to the body and thus deform with the body."}. We see, that this leads to the fact, that the local coordinate system is placed on $ \Gamma^{\gamma,M}$ and therefore $ \Omega^{M}$ becomes in this sense "master", as it is the body, where the local coordinate system and thought "observer" are placed.    
\\
\\
Now finally, we can formulate following contact boundary conditions, which are generally known as Hertz–Signorini–Moreau conditions for frictionless contact. They also have the structure of Kuhn-Tucker-Karush conditions, well known from the theory of optimization. According to \cite[Chapter 5, p.97]{Wriggers}, Hertz–Signorini–Moreau conditions are prescribed as
\beq\label{11}
g_{n}=\left(\mathbf{x}^{S}-\overline{\mathbf{x}}^{M}\right) \cdot \mathbf{n}^{M} \geq 0 \text{ on } \Gamma^{\gamma}
\eeq
\beq\label{12}
p_{n} \leq 0 \text{ on } \Gamma^{\gamma,C}
\eeq
\beq\label{13}
g_{n} p_{n}=0 \text{ on } \Gamma^{\gamma}
\eeq     
In Hertz–Signorini–Moreau conditions, $ g_{n} $ provides, as was explained above, measure of "gap" between corresponding master and slave points. $ p_{n} $ is normal component of $ \mathbf{t} $ defined as $ p_{n} = \mathbf{t}^{M}\cdot\mathbf{N}^{M} $, where $ \mathbf{t}^{M} = \mathbf{P}\cdot\mathbf{N}^{M} $ (due to action-reaction principle - Newton's third law - of course holds $ \mathbf{t}^{M} = -\mathbf{t}^{S}=\mathbf{P}\cdot\mathbf{N}^{S} $ so $ p_{n} = \mathbf{t}^{S}\cdot\mathbf{N}^{S} $). Condition on $ p_{n} $ requires, that the reaction force on the contact part of the boundary $ \Gamma^{\gamma,C}$ is none or compressive. Let us denote, that $ \mathbf{n} $ is normal to $\Gamma^{\gamma}$.
\subsection{Contact boundary value problem as variational inequality}
We will now formulate weak form of contact boundary value problem utilizing contact boundary conditions and weak form of problem of finite deformation of system of bodies $ \Omega^{\gamma} $ under static loading, which we both formulated in previous two sub-chapters.
\\
\\
Firstly, we will improve equation (\ref{3}) to hold also in case of contact on boundary $\Gamma^{\gamma,C}$. To simplify the notation of our problem, we will consider only two bodies in contact, so $\gamma = 2$. As there is a work applied on boundary $\Gamma^{\gamma,C}$, it even intuitively makes sense, that we have to somehow add to our equation some additional members to preserve equilibrium in system of bodies $\Omega^{\gamma}$. According to \cite[Chapter 4, p.137]{Yastrebov}, for case of normal frictionless contact, this could be done in following manner  
\bea\label{14}
&&\sum^{2}_{\gamma=1}\left(\int_{\Omega^{\gamma}} \mathbf{P} : \operatorname{Grad} \mathbf{v} \dd V-\int_{\Gamma_{\sigma}^{\gamma}} \overline{\mathbf{t}} \cdot \mathbf{v} \dd A-\int_{\Omega^{\gamma}} \rho_{0}\overline{\mathbf{b}} \cdot \mathbf{v} \dd V\right)+ \nonumber
\\ 
&&+\int_{\Gamma^{M,C}} p_{n}(v_{n}^{S}-\overline{v}_{n}^{M}) \dd A=0
\eea 
where $ v_{n}^{M} = \delta \left(\overline{\mathbf{x}}^{M} \cdot \mathbf{n}^{M}\right) = \delta \overline{\mathbf{x}}^{M} \cdot \mathbf{n}^{M} = \overline{\mathbf{v}}^{M} \cdot \mathbf{n}^{M} $ and $ v_{n}^{S} = \delta \left(\mathbf{x}^{S} \cdot \mathbf{n}^{S}\right) = \delta \mathbf{x}^{S} \cdot \mathbf{n}^{S} = \mathbf{v}^{S} \cdot \mathbf{n}^{S} $ (Operator $ \delta $ denotes Gateaux derivative). We can observe that the contact term in equation (\ref{14}) is represented by an integral over master surface $ \Gamma^{M,C} $. However, it is possible to set equivalent formulation based on integration over slave surface $ \Gamma^{S,C} $. This is a result of Newton's third law as $ \mathbf{t}^{M} = -\mathbf{t}^{S}=\mathbf{P}\cdot\mathbf{N}^{S} $ (see more details on derivation in \cite[Chapter 4, p.137-140]{Yastrebov}). Equation (\ref{14}) could be aso equivalently written as
\bea\label{15}
&&\sum^{2}_{\gamma=1}\left(\int_{\Omega^{\gamma}} \mathbf{P} : \operatorname{Grad} \mathbf{v} \dd V-\int_{\Gamma_{\sigma}^{\gamma}} \overline{\mathbf{t}} \cdot \mathbf{v} \dd A-\int_{\Omega^{\gamma}} \rho_{0}\overline{\mathbf{b}} \cdot \mathbf{v} \dd V\right)+ \nonumber
\\ 
&&+\int_{\Gamma^{M,C}} p_{n}\delta g_{n}\dd A=0
\eea
where $\delta g_{n}$ is a Gateaux derivative of gap function $g_{n}$. Now, to state the contact boundary value problem, we will utilize contact boundary conditions (\ref{11})-(\ref{13}). Firstly, from (\ref{11}) $g_{n}=\left(\mathbf{x}^{S}-\overline{\mathbf{x}}^{M}\right) \cdot \mathbf{n}^{M} \geq 0 $ we immediately see, that $\delta g_{n}=\left(\delta\mathbf{x}^{S}-\delta\overline{\mathbf{x}}^{M}\right) \cdot \mathbf{n}^{M} \geq 0 $. At the same time holds (\ref{12}) $ p_{n} \leq 0 $. These leads to 
\bea\label{16}
\int_{\Gamma^{M,C}} p_{n}\delta g_{n}\dd A \leq 0
\eea
and so
\bea\label{17}
&&\sum^{2}_{\gamma=1}\left(\int_{\Omega^{\gamma}} \mathbf{P} : \operatorname{Grad} \mathbf{v} \dd V-\int_{\Gamma_{\sigma}^{\gamma}} \overline{\mathbf{t}} \cdot \mathbf{v} \dd A-\int_{\Omega^{\gamma}} \rho_{0}\overline{\mathbf{b}} \cdot \mathbf{v} \dd V\right) \geq 0
\eea
If we add to inequality (\ref{17}) proper definition of sets of admissible functions
\begin{equation}\label{18}
\begin{array}{c}{\mathbb{V}=\left\{\mathbf{v} \in H^{1}(\Omega); \mathbf{v}=\mathbf{0} \text { on } \Gamma_{u}\right\}}
\\ 
{\mathbb{K}=\left\{\delta \mathbf{v} \in \mathbb{V}; \left(\mathbf{v}^{S}-\overline{\mathbf{v}}^{M}\right) \cdot \mathbf{n}^{M} \geq 0 \text { on } \Gamma^{\gamma,C}\right\}}\end{array} 
\end{equation}
we obtain variational inequality as stated in \cite{Duvaut} and \cite{Kikuchi}, where also the proof of equivalence with variational equality (\ref{15}) can be found. Here $H^{1}(\Omega)$ denotes Hilbert space of first order over domain $ \Omega = \Omega^{1} \cup \Omega^{2}$. This formulation is valid for any material, as it is based on "virtual work principle" which does not presume any material constitutive relation. However, instead of classical variational equations, different optimization methods are needed for numerical solution \cite[Chapter 4, p.142]{Yastrebov}. Details can be found in monographs \cite{Duvaut} and \cite{Kikuchi}. In \cite{Kikuchi}, the existence and uniqueness of the solution is proved for small deformation frictionless contact and questions of convergence of the finite element method are discussed here.
\\
\\
For hyperelastic material, problem of solution of variational inequality can be equivalently formulated \cite[Chapter 5, p.98]{Wriggers} as problem of minimization of functional of energy (analogically to (\ref{7})).
\begin{equation}\label{19}
\Pi(\bm{\varphi}) = \sum^{2}_{\gamma = 1}\left(\int_{\Omega^{\gamma}} W(\mathbf{C})\dd V-\int_{\Omega^{\gamma}} \rho_{0}\overline{\mathbf{b}} \cdot \mathbf{\bm{\varphi}} \dd V-\int_{\Gamma_{\sigma}^{\gamma}} \overline{\mathbf{t}} \cdot \mathbf{\bm{\varphi}} \dd A\right)=0
\end{equation}
subjected to constraint (\ref{11})
\beq
g_{n}=\left(\mathbf{x}^{S}-\overline{\mathbf{x}}^{M}\right) \cdot \mathbf{n}^{M} \geq 0 \text{ on } \Gamma^{\gamma} 
\nonumber
\eeq
If we consider the case of finite elasticity, the proof of existence of solution is given in \cite{Ciarlet} and \cite{Curnier}.
\subsection{Incremental contact boundary value problem as variational equality}
Nevertheless, instead of variational inequality approach, however interesting, we will focus in this thesis on methods based on variational equalities. To form the contact boundary value problem in form of "classical" variational equality, we need the knowledge of contact zone. This will be during numerical solution achieved by contact search algorithms (see \cite[Chapter 9]{Wriggers}). Contact search will be used during each step of the numerical solution, which enable us to form the contact boundary value problem incrementally as variational equality. As we will for the sake of simplicity consider the case of linear elasticity in following chapters, we will formulate the various formulations for numerical solution of contact boundary value problem on the basis of following energy functional, which is to be minimized
\bea\label{20}
\Pi&=&\sum^{n}_{\gamma = 1}\left(\int_{\Omega^{\gamma}} W(\mathbf{C})\dd V-\int_{\Omega^{\gamma}} \rho_{0}\overline{\mathbf{b}} \cdot \mathbf{\bm{\varphi}} \dd V-\int_{\Gamma_{\sigma}^{\gamma}} \overline{\mathbf{t}} \cdot \mathbf{\bm{\varphi}} \dd A\right) + \Pi_{C}=
\nonumber
\\
&=&\Pi(\bm{\varphi})+ \Pi_{C} = 0
\eea
Functional (\ref{20}) is written generally for the case of hyperelastic material and finite deformations. Let us remind, that $ \mathbf{\bm{\varphi}} $ was defined in (\ref{7}). $ \Pi_{C} $ is to be specified for every numerical solution method. Minimum of the functional $ \Pi $ will be searched threw condition, that Gateaux derivatives, with respect to all unknown functions $ \Pi $, must be zero valued. We intentionally omit the correct definition of function space on which we will seek the solution, as this will differ up to the method and definition of $ \Pi_{C} $. However, globally we seek fulfilling the contact boundary conditions (\ref{11})-(\ref{13}). We see, that implicitly, we will have to eventually numerically solve the variational equality similar to (\ref{15}). Yastrebov \cite[Chapter 4, p.144]{Yastrebov} mentions, that: "The variational inequality is hard to apply for contact with finite sliding and/or rotations. That is why, nowadays, most of the practical studies in contact mechanics are based on the
so-called variational equalities, which are easy to introduce in a finite element
framework and does not require totally new minimization techniques."   
\newpage
\section{Methods for "weak" imposition of contact boundary conditions in contact mechanics}
\subsection{Contact boundary value problem in linear elasticity}  
Let us remind, that we consider the case of frictionless normal contact. In the previous chapter, presuming we can identify the contact zone using search algorithms, we formulated contact boundary value problem for hyperelastic material as minimum of functional (\ref{7}). Despite we considered finite deformations in (\ref{7}) to present as general formulation of contact boundary value problem as possible, in this chapter we will limit ourselves only to case of linear elasticity (both materially and geometrically) to avoid complicated finite element discretized formulations. However, presented methods could be potentially extended to different material laws and finite deformations. Based on the notation from functional (\ref{7}), we will formulate our energy functional for linear elasticity contact problem as  
\bea\label{21}
\Pi&=&\sum^{n}_{\gamma = 1}\left(\int_{\Omega^{\gamma}} W(\mathbf{u})\dd V-\int_{\Omega^{\gamma}} \rho_{0}\overline{\mathbf{b}} \cdot \mathbf{u} \dd V-\int_{\Gamma_{\sigma}^{\gamma}} \overline{\mathbf{t}} \cdot \mathbf{u} \dd A\right) + \Pi_{C}=
\nonumber
\\
&=&\Pi(\mathbf{u})+ \Pi_{C} = 0
\eea  
Here $ \mathbf{u} = \mathbf{X} - \mathbf{x} $ and
\begin{equation}\label{22}
W(\mathbf{u})=\frac{1}{2} \bm{\sigma(\mathbf{u})} : \bm{\varepsilon(\mathbf{u})}
\end{equation} 
In equation (\ref{22}) and $ \bm{\sigma}(\mathbf{u}) $ is a Cauchy stress tensor, while
\beq\label{23}
\bm{\sigma}(\mathbf{u}) = \mathbf{D}_{e} \bm{\varepsilon}(\mathbf{u})
\eeq 
at the same time $ \bm{\varepsilon}(\mathbf{u}) $ is the Green-Lagrangian strain tensor for case of small deformation (see \cite[Chapter 2, p.24]{Wriggers}) and
\beq\label{24}
\bm{\varepsilon}(\mathbf{u}) = \operatorname{Grad}_{S}(\mathbf{u})=\frac{1}{2}\left(\operatorname{Grad}(\mathbf{u})+(\operatorname{Grad}(\mathbf{u}))^{T} \right) 
\eeq
Now, let us discuss the stationary condition of functional $ \Pi(\mathbf{u}) $ and its finite element disretization, as we will utilize it later. We will introduce the concept only briefly, interested reader could find details in \cite{Zienkiewicz} and \cite{Zienkiewicz2}. To state the problem properly, we can imagine, that $ \Pi_{C} = 0 $. As we still require $ \mathbf{u}=\overline{\mathbf{u}} \text{ on } \Gamma_{u}^{\gamma} $, we look for such 
\beq\label{25}
\mathbf{u}\in \mathbb{V}=\left\{\mathbf{v} \in H^{1}(\Omega); \mathbf{v}=\overline{\mathbf{v}} \text{ on } \Gamma_{u}^{\gamma}\right\}
\eeq
that $ \Pi(\mathbf{u}) $ is minimized. By requiring Gateaux derivative of $ \Pi(\mathbf{u}) $ to be zero, i.e.
\beq\label{26}
\delta \Pi(\mathbf{u},\mathbf{v})=\left.\frac{\dd}{\dd \alpha} \Pi(\mathbf{u}+\alpha \mathbf{v})\right|_{\alpha=0} = 0
\eeq
to be fulfilled for   
\beq\label{27}
\forall\mathbf{v}\in {\mathbb{V}=\left\{\mathbf{v} \in H^{1}(\Omega); \mathbf{v}=\mathbf{0} \text { on } \Gamma_{u}^{\gamma}\right\}} 
\eeq
we obtain the equation
\begin{equation}\label{28}
\sum^{n}_{\gamma = 1}\left(\int_{\Omega^{\gamma}} \bm{\sigma}(\mathbf{u}) : \bm{\varepsilon}(\mathbf{v}) \dd V\right)=\sum^{n}_{\gamma = 1}\left(\int_{\Omega^{\gamma}} \rho_{0}\overline{\mathbf{b}} \cdot \mathbf{v} \dd V+\int_{\Gamma_{\sigma}^{\gamma}} \overline{\mathbf{t}} \cdot \mathbf{\mathbf{v}} \dd A\right)
\end{equation}
again to be fulfilled for    
\beq
\forall\mathbf{v}\in {\mathbb{V}=\left\{\mathbf{v} \in H^{1}(\Omega); \mathbf{v}=\mathbf{0} \text { on } \Gamma_{u}^{\gamma}\right\}}
\nonumber 
\eeq
Now we will discretize our problem using finite element method. Our notation is mostly taken from \cite[p.886]{Jirasek}, where more details could be found. The discretization of $ \mathbf{u}^{e}=\left\{u^{e}_{1}, u^{e}_{2}, u^{e}_{3}\right\}^{T} $, where $ u^{e}_{i} $ are components of $ \mathbf{u}^{e} $ for element $ e $ with respect to Cartesian coordinates, is given as 
\begin{equation}\label{29}
u^{e}_{i}(\mathbf{x}) = \sum_{I=1}^{N_{n o d}} N_{I}(\mathbf{x}) d_{I i}, \quad i=1,2,3
\end{equation}
where $d_{I i}$ are unknown displacement parameters. $ N_{n o d} $ is a number of nods of the element $e$. We can introduce matrix notation now. Collecting displacement parameters into a column matrix $\mathbf{d}^{e}$ and the shape functions into a matrix $\mathbf{N}$, we can write
\beq\label{30}
\mathbf{u}^{e}(\mathbf{x}) = \mathbf{N}\mathbf{d}^{e}
\eeq
Despite we considered $ \bm{\varepsilon}(\mathbf{u}(\mathbf{x})) $ and $ \bm{\sigma}(\mathbf{u(\mathbf{x})}) $ as second-order tensors so far, for the purposes of finite element solution we will utilize so called Voight or engineering notation, i.e. $ \bm{\varepsilon}(\mathbf{u}(\mathbf{x})) $ and $ \bm{\sigma}(\mathbf{u}(\mathbf{x})) $ are considered as column matrices
\begin{equation}\label{31}
\bm{\sigma}(\mathbf{u}(\mathbf{x}))=\left\{\begin{array}{l}{\sigma_{x}} \\ {\sigma_{y}} \\ {\sigma_{z}} \\ {\tau_{y z}} \\ {\tau_{z x}} \\ {\tau_{x y}}\end{array}\right\}=\left\{\begin{array}{c}{\sigma_{11}} \\ {\sigma_{22}} \\ {\sigma_{33}} \\ {\sigma_{23}} \\ {\sigma_{31}} \\ {\sigma_{12}}\end{array}\right\},
\quad 
\bm{\varepsilon}(\mathbf{u}(\mathbf{x}))=\left\{\begin{array}{c}{\varepsilon_{x}} \\ {\varepsilon_{y}} \\ {\varepsilon_{z}} \\ {\gamma_{y z}} \\ {\gamma_{z x}} \\ {\gamma_{x y}}\end{array}\right\}=\left\{\begin{array}{c}{\varepsilon_{11}} \\ {\varepsilon_{22}} \\ {\varepsilon_{33}} \\ {2 \varepsilon_{23}} \\ {2 \varepsilon_{31}} \\ {2 \varepsilon_{12}}\end{array}\right\}
\end{equation}
Using this notation, we consider, that $\bm{\varepsilon}(\mathbf{u}(\mathbf{x}))=\bm{\partial} \mathbf{u}(\mathbf{x})$ and  $ \bm{\sigma}(\mathbf{u}(\mathbf{x}))=\mathbf{D}_{e} \mathbf{u}(\mathbf{x}) $, where $ \bm{\partial} $ denotes an operator matrix  
\begin{equation}\label{32}
\bm{\partial}=\left[\begin{array}{ccc}{\frac{\partial}{\partial x_{1}}} & {0} & {0} \\ {0} & {\frac{\partial}{\partial x_{2}}} & {0} \\ {0} & {0} & {\frac{\partial}{\partial x_{3}}} \\ {0} & {\frac{\partial}{\partial x_{3}}} & {\frac{\partial}{\partial x_{2}}} \\ {\frac{\partial}{\partial x_{3}}} & {0} & {\frac{\partial}{\partial x_{1}}} \\ {\frac{\partial}{\partial x_{2}}} & {\frac{\partial}{\partial x_{1}}} & {0}\end{array}\right]
\end{equation}
Let us also denote, that we still consider $\overline{\mathbf{t}}(\mathbf{x})=\mathbf{n} \bm{\sigma}(\mathbf{u}(\mathbf{x}))$ on $\Gamma^{\gamma}_{\sigma}$, where $\overline{\mathbf{t}}=\left\{\overline{t}_{1}, \overline{t}_{2}, \overline{t}_{3}\right\}^{T}$ and 
\begin{equation}\label{33}
\mathbf{n}=\left[\begin{array}{cccccc}{n_{1}} & {0} & {0} & {0} & {n_{3}} & {n_{2}} \\ {0} & {n_{2}} & {0} & {n_{3}} & {0} & {n_{1}} \\ {0} & {0} & {n_{3}} & {n_{2}} & {n_{1}} & {0}\end{array}\right]
\end{equation}
Now we can write 
\begin{equation}\label{34}
\bm{\varepsilon}^{e}(\mathbf{x})=\bm{\partial} \mathbf{u}^{e}(\mathbf{x}) = \bm{\partial} \mathbf{N}(\mathbf{x}) \mathbf{d}^{e}=\mathbf{B}(\mathbf{x}) \mathbf{d}^{e}
\end{equation}
\begin{equation}\label{35}
\bm{\sigma}^{e}(\mathbf{x}) = \mathbf{D}_{e}(\mathbf{x}) \mathbf{B}(\mathbf{x}) \mathbf{d}^{e}
\end{equation}
If we substitute relations (\ref{34}) and (\ref{35}) to (\ref{28}), we obtain
\begin{equation}\label{36}
\begin{aligned} 
\sum^{N_{elem}}_{e = 1}\sum^{n}_{\gamma = 1}&\left(\int_{\Omega^{\gamma}} \mathbf{d}^{eT} \mathbf{B}^{T}(\mathbf{x}) \mathbf{D}_{e}^{T}(\mathbf{x}) \mathbf{B}(\mathbf{x}) \mathbf{v}^{e} \dd V \right)=
\\
&=\sum^{N_{elem}}_{e = 1}\sum^{n}_{\gamma = 1}\left(\int_{\Omega^{\gamma}} \rho_{0}\overline{\mathbf{b}}^{eT} \mathbf{N}(\mathbf{x}) \mathbf{v}^{e} \dd V \right)
\\ 
&+\sum^{N_{elem}}_{e = 1}\sum^{n}_{\gamma = 1}\left(\int_{\Gamma^{\gamma}_{\sigma}} \overline{\mathbf{t}}^{eT} \mathbf{N}(\mathbf{x}) \mathbf{v}^{e} \dd A\right)
\end{aligned}
\end{equation}
In equation (\ref{36}) we consider, that we sum over all $ N_{elem} $ elements. 
Now, if we assemble local vectors $ \mathbf{d}^{e} $ and $ \mathbf{v}^{e} $ into global vector $ \mathbf{d}^{g} $ and $ \mathbf{v}^{g} $ using scaling matrix $ \mathbf{L}^{e} $ (see details in \cite{Zienkiewicz} and \cite{Zienkiewicz2}), we can write
\begin{equation}\label{37}
\begin{aligned} 
\mathbf{d}^{gT}\sum^{N_{elem}}_{e = 1}\mathbf{L}^{eT}\sum^{n}_{\gamma = 1}&\left(\int_{\Omega^{\gamma}} \mathbf{B}^{T}(\mathbf{x}) \mathbf{D}_{e}^{T}(\mathbf{x}) \mathbf{B}(\mathbf{x}) \dd V \right)\mathbf{L}^{e}\mathbf{v}^{g}=
\\
&=\sum^{N_{elem}}_{e = 1}\mathbf{L}^{eT}\sum^{n}_{\gamma = 1}\left(\int_{\Omega^{\gamma}} \rho_{0}\overline{\mathbf{b}}^{eT} \mathbf{N}(\mathbf{x}) \dd V \right)\mathbf{L}^{e}\mathbf{v}^{g}
\\ 
&+\sum^{N_{elem}}_{e = 1}\mathbf{L}^{eT}\sum^{n}_{\gamma = 1}\left(\int_{\Gamma^{\gamma}_{\sigma}} \overline{\mathbf{t}}^{eT} \mathbf{N}(\mathbf{x}) \dd A\right)\mathbf{L}^{e}\mathbf{v}^{g}
\end{aligned}
\end{equation}
If we denote
\beq\label{38}
\mathbf{K}^{T}=\sum^{N_{elem}}_{e = 1}\mathbf{L}^{eT}\sum^{n}_{\gamma = 1}\left(\int_{\Omega^{\gamma}} \mathbf{B}^{T}(\mathbf{x}) \mathbf{D}_{e}^{T}(\mathbf{x}) \mathbf{B}(\mathbf{x})\dd V \right)\mathbf{L}^{e}
\eeq
and
\bea\label{39}
\mathbf{f}^{T} &=& \sum^{N_{elem}}_{e = 1}\mathbf{L}^{eT}\sum^{n}_{\gamma = 1}\left(\int_{\Omega^{\gamma}} \rho_{0}\overline{\mathbf{b}}^{eT} \mathbf{N}(\mathbf{x}) \dd V \right)\mathbf{L}^{e}+
\\
&+&\sum^{N_{elem}}_{e = 1}\mathbf{L}^{eT}\sum^{n}_{\gamma = 1}\left(\int_{\Gamma^{\gamma}_{\sigma}} \overline{\mathbf{t}}^{eT} \mathbf{N}(\mathbf{x}) \dd A\right)\mathbf{L}^{e}
\nonumber
\eea
we can simply state our problem as
\begin{equation}\label{40}
\mathbf{d}^{gT} \mathbf{K}^{T} \mathbf{v}^{g}=\mathbf{f}^{T} \mathbf{v}^{g}
\end{equation}
and by transposition of all members of the equation we obtain
\begin{equation}\label{41}
\mathbf{v}^{gT} \mathbf{K}\mathbf{d}^{g}=\mathbf{v}^{gT}\mathbf{f} 
\end{equation}
which has to be fulfilled for 
\beq
\forall\mathbf{v}^{g}\in \left\{\mathbf{v}^{g} \in V^{h}(\Omega)\subset H^{1}(\Omega); \mathbf{v}^{g}=\mathbf{0} \text { on } \Gamma_{u}^{\gamma}\right\}
\footnote{$V^{h}$ is finite dimensional, for more details see \cite{Zienkiewicz}}
\nonumber 
\eeq
So it must hold
\begin{equation}\label{41}
\mathbf{K}\mathbf{d}^{g}=\mathbf{f} 
\end{equation}
\subsection{General view of methods for enforcing contact boundary conditions}
In this chapter we will introduce various methods for weak imposition of contact boundary conditions. We will present all these methods on linear elastic contact problem discussed in the previous chapter, where we derived functional of energy (\ref{21}), which reads
\bea
\Pi&=&\sum^{2}_{\gamma = 1}\left(\int_{\Omega^{\gamma}} W(\mathbf{u})\dd V-\int_{\Omega^{\gamma}} \rho_{0}\overline{\mathbf{b}} \cdot \mathbf{u} \dd V-\int_{\Gamma_{\sigma}^{\gamma}} \overline{\mathbf{t}} \cdot \mathbf{u} \dd A\right) + \Pi_{C}=
\nonumber
\\
&=&\Pi(\mathbf{u})+ \Pi_{C} = 0
\nonumber
\eea   
Here, $ \Pi(\mathbf{u}) $ denotes the total potential energy, which is associated with deformation of bodies $\Omega^{\gamma}$ excluded the potential energy connected to contact constraints. This is specified by $ \Pi_{C} $, which can be also understood as expression enforcing the contact boundary conditions on $\Gamma^{C,\gamma}$. For the sake of simplicity of the notation we consider only two bodies in contact.  
\\
\\
We will use this framework to present various methods for "weak" imposition of contact boundary conditions, which will differ in definition of $ \Pi_{C} $. We can observe, that we are basically enforcing the contact boundary conditions by incorporating them into the functional of total potential energy. Even intiutively we feel, that they are so enforced in somehow "weak" sense. This is because the stationary conditions of the functional $ \Pi $ eventually leads to "weak" form of equilibrium equations of continua. Because the contact boundary conditions can be understood as Dirichlet boundary conditions \cite[Chapter 4, p.109]{Yastrebov}, enforcing contact boundary conditions by incorporating them into the energy functional can be also understood as "weak imposition of Dirichlet boundary conditions" \cite{Augarde}. Despite the fact, that we will focus our work on methods for "weak" methods, in some cases it is also possible to impose them in "strong" sense. Both "weak" and "strong" methods also differ according to whether the bodies are discretized in finite element methods as matching or non-matching mesh.
\\
\\
Let us start with the case, that bodies $\Omega^{\gamma} $ are discretized to form so called matching meshes. In this case, the nodes of the mesh assume the same coordinates, when in contact. We will present two classical "weak" methods, which are applicable for matching meshes - Lagrange multiplier method and Penalty method. There exist of course many other methods, which are basically combinations of these two approaches, such as the Augmented Lagrange method and Perturbed Lagrange method (see \cite[Chapter 5, p.100]{Wriggers} for more details) which can be advantageous in various cases. It can also seem intriguing, why we are trying to apply boundary conditions in "weak" manner instead of directly applying the contact boundary condition (\ref{11})
\beq
g_{n}=\left(\mathbf{x}^{S}-\overline{\mathbf{x}}^{M}\right) \cdot \mathbf{n}^{M} = 0 \text{ on } \Gamma^{\gamma,C}
\nonumber
\eeq  
to the system of equations gained by discretization of the problem by finite element method. That is a standard practice in finite element analysis and leads to the reduction of the number of the unknowns of the problem. This is theoretically possible to do even in case of discretized contact boundary value problem, Wriggers et al \cite[Chapter 5, p.104]{Wriggers} calls this procedure "Direct constraint elimination", Zienkiewicz \cite[Chapter 7, p.195]{Zienkiewicz2} just "Constraint elimination". However, as Wriggers et al. \cite[Chapter 5, p.104]{Wriggers} mentions: "An efficient enforcement of the constraint depends heavily upon the discretization." and Zienkiewicz et al. \cite[Chapter 7, p.195]{Zienkiewicz2} comments: "In the simple frictionless node-to-node contact case it is simple to implement as no transformations are needed to write the constraint equation. In a general case, however, the approach can become quite cumbersome and it is often simpler to use the Lagrange multiplier form directly or to consider other related approaches." Therefore, those are the reasons, why we shall focus on methods providing weak imposition of boundary conditions rather than "Direct constraint elimination".  
\\
\\
However, especially in case of geometrically complex structures, generation of mesh is non-trivial question. This leads to the generation of complex unstructured meshes, which are often non-matching, i.e. nodes in contact do not assume the same coordinates. This could be also the case of parallel computations, where domains with non-matching grids are often constructed (as \cite[Chapter 7, p.183]{Wriggers} mentions). In case of non-matching meshes, none of the previously mentioned strategies could be straightforwardly used, even in the geometrically linear case, which we consider. Instead, few approaches are possible. One is based on so-called contact segments introduced by Simo et al \cite{Simo}.  Wriggers et al. (as \cite[Chapter 7, p.183]{Wriggers} mentions) recommends using contact segments together with Perturbed Lagrange method. Another approach present so-called Mortar methods. Wriggers et al. in \cite[Chapter 7, p.188-189]{Wriggers} differentiate two sub-types of this method. First one, described by \cite{Bernadi}and \cite{Wohlmuth}, is based on direct enforcement of the constraints, and therefore is basically equivalent to the direct constraint elimination as described before. Wriggers er al. \cite[Chapter 7, p.189]{Wriggers} mentions, that it leads to a positive definite system of equations. The other sub-type of mortar methods is connected with 'weak' enforcement of the constraints utilizing Lagrange multiplier method (see \cite{Wohlmuth}). Another possible method for non-matching meshes is Nitsche's method, which enforces contact boundary conditions in "weak" sense. On this method, we will particularly concentrate.
\\
\\
Before describing some of the methods for "weak" imposition of boundary conditions in more detail, we have to mention, that from the mathematical point of view, all these methods lead to saddle point problems. It means the found stationary point of the functional, even if it exists and is unique, is not a minimum or maximum but a saddle point. In order to provide, that solution of our saddle point problem exists and is uniqe, and implicitly that our finite element discretization is stable, we have to guarantee, that Babuška-Brezzi or so called "sup-inf" condition is satisfied. Literature overview and the underlying mathematical theory of Babuška-Brezzi condition in context of contact boundary value problem is given in \cite{Wohlmuth} and \cite{Brezzi}.  
\subsection{The Lagrange Multiplier Method}
Theory of this method, in terms of using it in finite element method for fulfilling Dirichlet boundary conditions, was first derived by Babuška \cite{Babuska1}. Let us remind, that we consider the case of frictionless contact. Sheerly for the simplicity of the notation, we also consider only two bodies $ \Omega^{\gamma} $ in contact. The Lagrange Multiplier Method defines $ \Pi_{C} $ as (see \cite[Chapter 5, p.100-101]{Wriggers}, \cite[Chapter 7, p.162-165]{Wriggers}, \cite[Chapter 5, p.157-170]{Yastrebov} and \cite{Augarde})
\beq\label{43}
\Pi_{C} = \int_{\Gamma^{M,C}} \lambda_{n} g_{n}\dd A
\eeq
where, as $ \mathbf{x} = \mathbf{X} + \mathbf{u} $ (as introduced in chapter 1), holds
\bea\label{44}
g_{n}&=&\left(\mathbf{x}^{S}-\overline{\mathbf{x}}^{M}\right) \cdot \mathbf{n}^{M} = \left(\mathbf{u}^{S}-\overline{\mathbf{u}}^{M}\right) \cdot \mathbf{n}^{M} + \left(\mathbf{X}^{S}-\overline{\mathbf{X}}^{M}\right) \cdot \mathbf{n}^{M}
\nonumber 
\\
&=& \left(\mathbf{u}^{S}-\overline{\mathbf{u}}^{M}\right) \cdot \mathbf{n}^{M} + g_{0}
\eea  
As for system of two bodies $ \Pi(\lambda,\mathbf{u}) $ is defined as (see (\ref{21}))   
\bea\label{45}
\Pi(\lambda,\mathbf{u})&=&\sum^{2}_{\gamma = 1}\left(\frac{1}{2} \int_{\Omega^{\gamma}} \bm{\sigma(\mathbf{u})} : \bm{\varepsilon(\mathbf{u})}\dd V-\int_{\Omega^{\gamma}} \rho_{0}\overline{\mathbf{b}} \cdot \mathbf{u} \dd V-\int_{\Gamma_{\sigma}^{\gamma}} \overline{\mathbf{t}} \cdot \mathbf{u} \dd A\right) +
\nonumber
\\
&+& \Pi_{C}=\Pi(\mathbf{u})+ \Pi_{C}(\lambda,\mathbf{u}) = 0
\nonumber
\eea
the stationary condition of functional $ \Pi(\lambda,\mathbf{u}) $ 
\beq\
\delta \Pi(\lambda,\delta \lambda,\mathbf{u},\mathbf{v})=\left.\frac{\partial}{\partial \alpha\partial\beta}\Pi(\lambda+\beta \delta \lambda,\mathbf{u}+\alpha \mathbf{v})\right|_{\alpha=0,\beta=0} = 0
\eeq
Gives us following variational equality. We utilize here the stationary condition of $\Pi(\mathbf{u})$ derived in the previous sub-chapter (see (\ref{28}))   
\bea\label{46}
\begin{aligned}
\sum^{2}_{\gamma = 1}&\left(\int_{\Omega^{\gamma}} \bm{\sigma}(\mathbf{u}) : \bm{\varepsilon}(\mathbf{v}) \dd V-\int_{\Omega^{\gamma}} \rho_{0}\overline{\mathbf{b}} \cdot \mathbf{v} \dd V-\int_{\Gamma_{\sigma}^{\gamma}} \overline{\mathbf{t}} \cdot \mathbf{\mathbf{v}} \dd A\right)+
\\
&+\int_{\Gamma^{M,C}} \delta\lambda_{n} g_{n}\dd A+\int_{\Gamma^{M,C}} \lambda_{n} \delta g_{n}\dd A = 0
\end{aligned}
\eea
to be fulfilled for    
\beq\label{47}
\begin{array}{c}
{\forall\mathbf{v}\in {\mathbb{V}=\left\{\mathbf{v} \in H^{1}(\Omega); \mathbf{v}=\mathbf{0} \text { on } \Gamma_{u}^{\gamma}\right\}}}
\\
{\lambda_{n} \leq 0 \text{ on } \Gamma^{\gamma,C}}
\end{array}
\eeq
In (\ref{46}), $ \delta\lambda_{n}  $ and $ \delta g_{n}  $ denotes Gateaux derivative of $ \lambda_{n}  $ and $ g_{n}  $ respectively. $ \lambda_{n} $ can be identified (see \cite[Chapter 5, p.100-101]{Wriggers}) as $ p_{n} $.  We remind that $ p_{n} = \mathbf{t}^{M}\cdot\mathbf{n}^{M} $ and $ \mathbf{t}^{M} =\bm{\sigma}\cdot\mathbf{n}^{M} $ as in case of linear elasticity is considered $ \mathbf{n}^{M} \approx \mathbf{N}^{M} $.
Now we shall concentrate on finite element discretization of (\ref{46}). If we approximate $\lambda$ as 
\begin{equation}\label{48}
\lambda^{e}(\mathbf{x}) = \sum_{I=1}^{N_{n o d}} N_{I}^{\lambda}(\mathbf{x}) \lambda_{I} = \mathbf{N}^{\lambda}\bm{\lambda}^{e}
\end{equation}
We also consider, as in \cite[Chapter 7, p.165]{Wriggers}, that we can express $ g_{n}^{e}(\mathbf{u}) = \mathbf{G}\mathbf{d}^{e} $. This is possible, because $g_{n}$ depends on $\mathbf{u}$, as $ g_{n}=\left(\mathbf{u}^{S}-\overline{\mathbf{u}}^{M}\right) \cdot \mathbf{n}^{M} + g_{0} $. Now, we will utilize results for discerization of weak form of contact-less linear elastic problem presented in sub-chapter 3.1, where we obtained discretization of (\ref{28}). With all these, we get finite element discretization of (\ref{46}) in following form           
\begin{equation}\label{49}
\left[\begin{array}{ll}{\mathbf{K}} & {\mathbf{M}} \\ {\mathbf{M}^{T}} & {\mathbf{0}}\end{array}\right]\left\{\begin{array}{l}{\mathbf{d}^{g}} \\ {\bm{\lambda}^{g}}\end{array}\right\}=\left\{\begin{array}{l}{\mathbf{f}} \\ {\mathbf{0}}\end{array}\right\}
\end{equation} 
where
\begin{equation}\label{50}
\mathbf{M}=\sum^{N_{elem}}_{e = 1}\int_{\Gamma^{M,C}} \mathbf{N}^{\lambda,T} \mathbf{G} \dd A
\end{equation}  
It should be point out, that the choice of approximation functions, especially for $ \mathbf{N}^{\lambda} $ is a non-trivial question, as it must be done in a way to satisfy mentioned Babuška-Brezzi condition to achieve stability of our discretization.  
If approximation functions are chosen poorly (Which, as \cite{Augarde} mentions, is the case of most naive choices), poor performance of the method was observed (\cite{Augarde}), as oscillation and locking occurs. Another problems of the method are, that the dimension of the final system of equations increases compared to dimension of $ \mathbf{K} $ and resulting stiffness matrix is generally not positive definite of banded. As \cite{Augarde} mentions in more detail, various approaches has been developed to overcome this issue. \cite{Augarde} on the other hand argues, that the Lagrange Multiplier Method is very straightforward and applicable to broad scale of problems.   
\subsection{The Penalty Method}
This method was also analyzed in terms of applications to finite element method by Babuška \cite{Babuska2}, including error estimates and rate of convergence. This method defines $\Pi_{C}$ as 
\beq\label{51}
\Pi_{C} = \frac{1}{2}\int_{\Gamma^{M,C}} \epsilon g^{2}_{n}\dd A, \quad \epsilon > 0
\eeq   
Similarly to the case of the Lagrange Multiplier Method, stationary condition of functional $\Pi (\mathbf{u})$\footnote{Except $\Pi (\mathbf{u})$ now does not depend on Lagrange Multiplier $ \lambda $} gives us following variational equation
\bea\label{52}
\begin{aligned}
\sum^{2}_{\gamma = 1}&\left(\int_{\Omega^{\gamma}} \bm{\sigma}(\mathbf{u}) : \bm{\varepsilon}(\mathbf{v}) \dd V-\int_{\Omega^{\gamma}} \rho_{0}\overline{\mathbf{b}} \cdot \mathbf{v} \dd V-\int_{\Gamma_{\sigma}^{\gamma}} \overline{\mathbf{t}} \cdot \mathbf{\mathbf{v}} \dd A\right)+
\\
&+\int_{\Gamma^{M,C}} \epsilon g_{n} \delta g_{n} \dd A = 0
\end{aligned}
\eea
to be fulfilled for    
\beq\label{53}
\begin{array}{c}
{\forall\mathbf{v}\in {\mathbb{V}=\left\{\mathbf{v} \in H^{1}(\Omega); \mathbf{v}=\mathbf{0} \text { on } \Gamma_{u}^{\gamma}\right\}}}
\\
{\epsilon > 0}
\end{array}
\eeq
Now we could investigate, whether this method really leads to fulfilment of contact boundary conditions in strong sense. If we apply Gauss-Ostrogradsky theorem to first member of (\ref{52}), we get\footnote{To simplify the notation, we will utilize, that $  \bm{\sigma}(\mathbf{u}) : \bm{\nabla}_{S}\mathbf{v} = \bm{\sigma}(\mathbf{u}) : \bm{\nabla}\mathbf{v} $ as $\bm{\sigma}(\mathbf{u})$ is a symmetric second-order tensor}
\bea\label{54}
\begin{aligned}
&\sum^{2}_{\gamma = 1}\int_{\Omega^{\gamma}} \bm{\sigma}(\mathbf{u}) : \bm{\varepsilon}(\mathbf{v}) \dd V = \sum^{2}_{\gamma = 1}\int_{\Omega^{\gamma}} \bm{\sigma}(\mathbf{u}) : \bm{\nabla}_{S}\mathbf{v} \dd V =\sum^{2}_{\gamma = 1}\int_{\Omega^{\gamma}} \bm{\sigma}(\mathbf{u}) : \bm{\nabla}\mathbf{v} \dd V =
\\
&=\sum^{2}_{\gamma = 1}\left(-\int_{\Omega^{\gamma}}\left(\bm{\nabla}\cdot\bm{\sigma}(\mathbf{u})\right) \cdot \mathbf{v} \dd V + \int_{\Gamma^{\gamma}_{\sigma}} \left(\mathbf{n}\cdot\bm{\sigma(\mathbf{u})} \right)\cdot\mathbf{v}\dd A\right)-\int_{\Gamma^{M,C}} p_{n} \delta g_{n} \dd A
\end{aligned}
\eea 
If we will suppose now, that we can choose test functions $\mathbf{v}$ in such a way, that they are non-zero on $ \Gamma^{M,C} $ and zero on $ \Gamma^{M,\gamma}-\Gamma^{M,C} $ (which is really possible, see \cite{Augarde}, though it has to be correctly proofed), then we see, that we can obtain 'strong' contact boundary condition which reads
\beq\label{55}
-p_{n}+\epsilon g_{n} = 0
\eeq
which, because we presume $\epsilon > 0$ can be rewritten as   
\beq\label{56}
g_{n} = \frac{1}{\epsilon} p_{n}
\eeq
We see, that form the mechanical point of view, this condition is inconsistent with contact boundary condition $ g_{n} = 0 $, which we originally required. It could be easily seen, that $ \epsilon \rightarrow \infty $, than $g_{n} \rightarrow 0 $ in (\ref{55}). Though this is an intuitive observation rather than theoretical proof, it was shown (see for example \cite{Luenberger}), that the solution of the Lagrange Multiplier Method is equivalent to the Penalty method for $ \epsilon \rightarrow \infty $. It is therefore clear, that to achieve good approximation, we have to choose value of $\epsilon$ high enough. The problem is, that it could be complicated to choose the appropriate value of $\epsilon$, which is often guessed on the basis of empirical estimate \cite{Augarde}. From this it could seem, that higher $\epsilon$ leads to better approximation of the solution. However, if $\epsilon$ is too high, it could lead to ill-conditioned system of linear equations after finite element discretization. So, in practical computations, as \cite[Chapter 5, p.102]{Wriggers} mentions, the value of          $\epsilon$ is chosen such, that it avoids ill-conditioning of the discretized problem. 
\\
\\
Now let us focus on finite element discretization of the problem. As in the previous sub-chapter, we assume, that $ g_{n}^{e}(\mathbf{u}) = \mathbf{G}\mathbf{d}^{e} $. Therefore we obtain finite element discertization in form 
\begin{equation}\label{57}
\left(\mathbf{K}+\mathbf{K}_{p}\right) \mathbf{d}^{g}=\mathbf{f}
\end{equation}
where
\begin{equation}\label{58}
\mathbf{K}_{p}=\sum^{N_{elem}}_{e = 1}\int_{\Gamma^{M,C}} \mathbf{G}^{T} \mathbf{G} \dd A
\end{equation} 
In comparison with the Lagrange Multiplier Method, the Penalty Method does not add additional unknown variables $ \bm{\lambda}^{g} $ to the problem and so the dimension stiffness matrix does not increase compared to $ \mathbf{K} $. Compared to the Lagrange Multiplier method, the resulting stiffness matrix is symmetric and positive definite if $ \epsilon $ is high enough. The problem of this method, as we discussed previously, is that $ \epsilon $ has to be chosen empirically large enough to approximate the solution accurately but not too large, to give rise to the ill-conditioning of the stiffness matrix \cite{Augarde}.
\\
\\
It is also worth mentioning, that though introducing the penalty term is mathematically motivated, it also has a physical interpretation. It could be roughly understood, as if the constitutive equation was introduced in the boundary, it means that interface between two bodies is not relatively rigid, but is rather represented by non-linear springs. $\epsilon$ than can be seen as stiffness parameter.   
\subsection{The Nitsche's method}
This method was introduced by Nitsche in 1970s \cite{Nitsche} and in some terms it contains features both of the Lagrange Multiplier Method and The Penalty Method. As \cite{Augarde} states, Nitsche's method was also used to deal with decomposed domains in isogeometric analysis \cite{Apostolatos}, \cite{Ruess} and and discontinuous elements in the discontinuous Galerkin method \cite{Hansbo}. Nitsche's method introduces $\Pi_{C}$ as 
\beq\label{59}
\Pi_{C} = \int_{\Gamma^{M,C}} p_{n} g_{n}\dd A+\frac{1}{2}\int_{\Gamma^{M,C}} \epsilon g^{2}_{n}\dd A
\eeq   
As in the case of previously mentioned methods, stationary condition of functional $\Pi (\mathbf{u})$ gives us following variational equation
\bea\label{60}
\begin{aligned}
\sum^{2}_{\gamma = 1}&\left(\int_{\Omega^{\gamma}} \bm{\sigma}(\mathbf{u}) : \bm{\varepsilon}(\mathbf{v}) \dd V-\int_{\Omega^{\gamma}} \rho_{0}\overline{\mathbf{b}} \cdot \mathbf{v} \dd V-\int_{\Gamma_{\sigma}^{\gamma}} \overline{\mathbf{t}} \cdot \mathbf{\mathbf{v}} \dd A\right)+
\\
&+\int_{\Gamma^{M,C}} \delta p_{n} g_{n}\dd A+\int_{\Gamma^{M,C}} p_{n} \delta g_{n}\dd A+\int_{\Gamma^{M,C}} \epsilon g_{n} \delta g_{n} \dd A = 0
\end{aligned}
\eea
to be fulfilled for    
\beq\label{61}
\begin{array}{c}
{\forall\mathbf{v}\in {\mathbb{V}=\left\{\mathbf{v} \in H^{1}(\Omega); \mathbf{v}=\mathbf{0} \text { on } \Gamma_{u}^{\gamma}\right\}}}
\\
{\epsilon > 0}
\end{array}
\eeq 
Now, similarly as for the Penalty Method, let us focus on whether this method leads to fulfilment of contact boundary conditions in strong sense. Again, if we apply Gauss-Ostrogradsky theorem to first member of (\ref{52}), we get\footnote{To simplify the notation, we will utilize, that $  \bm{\sigma}(\mathbf{u}) : \bm{\nabla}_{S}\mathbf{v} = \bm{\sigma}(\mathbf{u}) : \bm{\nabla}\mathbf{v} $ as $\bm{\sigma}(\mathbf{u})$ is a symmetric second-order tensor}
\bea\label{62}
\begin{aligned}
&\sum^{2}_{\gamma = 1}\int_{\Omega^{\gamma}} \bm{\sigma}(\mathbf{u}) : \bm{\varepsilon}(\mathbf{v}) \dd V = \sum^{2}_{\gamma = 1}\int_{\Omega^{\gamma}} \bm{\sigma}(\mathbf{u}) : \bm{\nabla}_{S}\mathbf{v} \dd V =\sum^{2}_{\gamma = 1}\int_{\Omega^{\gamma}} \bm{\sigma}(\mathbf{u}) : \bm{\nabla}\mathbf{v} \dd V =
\\
&=\sum^{2}_{\gamma = 1}\left(-\int_{\Omega^{\gamma}}\left(\bm{\nabla}\cdot\bm{\sigma}(\mathbf{u})\right) \cdot \mathbf{v} \dd V + \int_{\Gamma^{\gamma}_{\sigma}} \left(\mathbf{n}\cdot\bm{\sigma(\mathbf{u})} \right)\cdot\mathbf{v}\dd A\right)-\int_{\Gamma^{M,C}} p_{n} \delta g_{n} \dd A
\end{aligned}
\eea 
Supposing we can choose test functions $\mathbf{v}$ in such a way, that they are non-zero on $ \Gamma^{M,C} $ and zero on $ \Gamma^{M,\gamma}-\Gamma^{M,C} $ we obtain 'strong' contact boundary condition
\beq\label{63}
\int_{\Gamma^{M,C}} -p_{n} \delta g_{n} +\delta p_{n} g_{n}+ p_{n} \delta g_{n}+\epsilon g_{n} \delta g_{n} \dd A = 0
\eeq
which is equivalent to    
\beq\label{64}
\int_{\Gamma^{M,C}} g_{n}\left(\epsilon  \delta g_{n}+\delta p_{n}\right) \dd A = 0
\eeq 
As $ \delta g_{n} $, $ \delta p_{n} $ can be understood as arbitrarily small infinitesimal values and it is assumed, that $\epsilon > 0$, we can observe, that this condition leads for arbitrarily positive value of $ \epsilon $ to the exact fulfilment of condition $ g_{n} = 0 $. Again, this is rather an observation than proof, but the independence on penalty parameter $\epsilon $ could be rigorously proofed. See that the penalty term is added to functional of total potential energy  (\ref{59}) and so weak formulation (\ref{60}) to avoid ill-conditioning of the linear equation system resulting from discretization of our problem \cite[Chapter 5, p.106]{Wriggers}. The \cite{Augarde} states, that Nitsche's method gives conditionally stable set of linear equations after discretization. \cite{Augarde} also provides estimates of $\epsilon $ based on largest eigenvalue of specified matrix eigenvalue problem. 
\\
\\
Finite element discretization could be found in \cite[Chapter 7, p.194-198]{Wriggers}. Advantage of Nitesche's method is, that no additional variables are introduced to the resulting system of linear equations, as is the case for the Lagrange Multiplier method. At the same time, the gained system of linear equations is symmetric and the penalty parameter can be properly estimated in advance. \cite{Augarde} mentions, that estimate procedure does not bring too much extra cost to computation. The problem of the method is similar to the Penalty Method, as if the value of penalty parameter $\epsilon$ is too high, the resulting stiffness matrix could be ill-conditioned. \cite{Augarde} states, that if the method is applied together with the discontinuous Galerkin method: "Additional calculations are performed which are costly in each element if parameter estimation is needed". \cite[Chapter 6, p.83]{Izi} states, that: "...the necessity to implement the whole structural finite element (because of $p_{n}$) positions the Nitsche's method among the very seldomly used methods in practical FE codes."  
     

 

\newpage
\begin{thebibliography}{99}
\addcontentsline{toc}{section}{References}
\bibitem{Wriggers} Wriggers, P. Computational Contact Mechanics. Chichester: Wiley, 2002.

\bibitem{Izi} Konyukhov, Alexander, and Ridvan Izi. Introduction to Computational Contact Mechanics : A Geometrical Approach. Wiley Series in Computational Mechanics. 2015.

\bibitem{Augarde} Lu, Kaizhou, Charles E. Augarde, William M. Coombs, and Zhendong Hu. ‘Weak Impositions of Dirichlet Boundary Conditions in Solid Mechanics: A Critique of Current Approaches and Extension to Partially Prescribed Boundaries’. Computer Methods in Applied Mechanics and Engineering 348 (1 May 2019): 632–59. https://doi.org/10.1016/j.cma.2019.01.035.

\bibitem{Yastrebov} Yastrebov, Vladislav A. Numerical Methods in Contact Mechanics. Numerical Methods in Engineering Series. London : Hoboken, N.J.: ISTE : Wiley, 2013.

\bibitem{Duvaut} Duvaut G., Lions J.L., “Elasticité avec frottement”, Journal de Mécanique,vol. 10, pp. 409–420, 1971.

\bibitem{Kikuchi} Kikuchi N., Oden J.T., Contact Problems in Elasticity: A Study of Variational Inequalities and Finite Element Methods, SIAM, Philadelphia, 1988.

\bibitem{Ciarlet} Ciarlet P.G. Mathematical Elasticity I: Three-dimensional Elasticity. North-Holland, Amsterdam, 1988

\bibitem{Curnier} Curnier A., He Q.C. and Telega J.J. Formulation of unilateral contact between two elastic bodies undergoing finite deformation. C. R. Acad. Sci. Paris,
314:1–6, 1992

\bibitem{Jirasek} Jirásek, Milan. Basic concepts and equations of solid mechanics. Revue européenne de génie civil (2007). 11. 10.3166/regc.11.879-892.

\bibitem{Zienkiewicz} Zienkiewicz, Taylor, Zhu, Taylor, Robert L., Zhu, J. Z., and Zienkiewicz, O. C. The Finite Element Method : Its Basis and Fundamentals. Sixth ed. Finite Element Method Set, 2005

\bibitem{Zienkiewicz2} Zienkiewicz, Taylor, Taylor, Robert L., and Zienkiewicz, O. C. The Finite Element Method for Solid and Structural Mechanics. Sixth ed. Finite Element Method Set, 2005

\bibitem{Simo} Simo J.C. A finite strain beam formulation. The three-dimensional dynamic
problem. Part I. Computer Methods in Applied Mechanics and Engineering,
49:55–70, 1985

\bibitem{Bernadi} Bernadi C., Maday Y. and Patera A. A new nonconforming approach to
domain decomposition: the motar element method. In H. Brezis and J.L.
Lions, editors, Nonlinear Partial Differential Equations and their Application.
Pitman, 1994

\bibitem{Wohlmuth} Wohlmuth B.I. Discretization Methods and Iterative Solvers based on Domain Decomposition. Springer Verlag, Berlin, Heidelberg, New York, 2000

\bibitem{Brezzi} Boffi, Daniele, Franco Brezzi, and Michel Fortin. Mixed Finite Element Methods and Applications. Springer Series in Computational Mathematics. Berlin Heidelberg: Springer-Verlag, 2013. https://www.springer.com/gb/book/9783642365188.

\bibitem{Babuska1} I. Babuška, The finite element method with lagrangian multipliers, Numer. Math. 20 (3) (1973) 179–192.

\bibitem{Babuska2} I. Babuška, The finite element method with penalty, Math. Comp. 27 (122) (1973) 221–228.

\bibitem{Luenberger} Luenberger D.G. Linear and Nonlinear Programming. Addison-Wesley, Reading, Mass., second edition (1984)

\bibitem{Nitsche} J. Nitsche, Über ein variationsprinzip zur lösung von Dirichlet-problemen bei verwendung von teilräumen, die keinen randbedingungen
unterworfen sind, in: Abhandlungen aus dem mathematischen Seminar der Universität Hamburg, vol. 36, Springer, Elsevier, 1971, pp. 9–15.

\bibitem{Apostolatos} A. Apostolatos, R. Schmidt, R. Wüchner, K.-U. Bletzinger, A Nitsche-type formulation and comparison of the most common domain
decomposition methods in isogeometric analysis, Internat. J. Numer. Methods Engrg. 97 (7) (2014) 473–504

\bibitem{Ruess} M. Ruess, D. Schillinger, A.I. Özcan, E. Rank, Weak coupling for isogeometric analysis of non-matching and trimmed multi-patch
geometries, Comput. Methods Appl. Mech. Engrg. 269 (2014) 46–47

\bibitem{Hansbo} P. Hansbo, M.G. Larson, Discontinuous Galerkin methods for incompressible and nearly incompressible elasticity by Nitsche’s method,
Comput. Methods Appl. Mech. Engrg. 191 (17–18) (2002) 1895–1908.

 
\end{thebibliography}
 
\end{document}

%notes

and $ \Gamma^{\gamma,C} \cap \left(\Gamma^{\gamma}_{u} \cup \Gamma^{\gamma}_{\sigma} %\right) = \emptyset $